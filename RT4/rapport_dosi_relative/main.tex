\documentclass{article}
\usepackage[utf8]{inputenc}
\usepackage{amsmath}
\usepackage{amsfonts}
\usepackage{esint}
\usepackage{geometry}
\usepackage{color}
\usepackage{fancyhdr}
\usepackage{ctable}
\usepackage{fancybox}
\usepackage{tabularx}
\usepackage{array}
\usepackage{booktabs}
\usepackage{subcaption}
\usepackage[french]{babel}
\usepackage{dsfont}
\usepackage{setspace}
\usepackage[french]{minitoc}
\usepackage{multicol}
\usepackage{multirow}
\usepackage[hidelinks]{hyperref}
\usepackage{graphicx}
\usepackage[T1]{fontenc}
\usepackage{xcolor}
\usepackage{listings}

\geometry{top=2.3cm, bottom=2.3cm, left=2.5cm, right=2.5cm}

\addtocounter{tocdepth}{3}
\setcounter{secnumdepth}{3}


\definecolor{codegreen}{rgb}{0,0.6,0}
\definecolor{codegray}{rgb}{0.5,0.5,0.5}
\definecolor{codepurple}{rgb}{0.58,0,0.82}
\definecolor{backcolour}{rgb}{0.95,0.95,0.92}

\lstdefinestyle{mystyle}{
  backgroundcolor=\color{white}, commentstyle=\color{codegreen},
  keywordstyle=\color{magenta},
  numberstyle=\tiny\color{codegray},
  stringstyle=\color{codepurple},
  basicstyle=\ttfamily\footnotesize,
  breakatwhitespace=false,         
  breaklines=true,                 
  captionpos=b,                    
  keepspaces=true,                 
  numbers=left,                    
  numbersep=5pt,                  
  showspaces=false,                
  showstringspaces=false,
  showtabs=false,                  
  tabsize=2
}

\lstset{style=mystyle}

\begin{document}

%%%%%%%%%%%%%%%%%%%%%%%%%%%%%%%%%%%%%%%%%%%%%%%%%%%%%%
%%%%%%%%%%%%%%%%%%%% PRÉSENTATION %%%%%%%%%%%%%%%%%%%%
%%%%%%%%%%%%%%%%%%%%%%%%%%%%%%%%%%%%%%%%%%%%%%%%%%%%%%

\begin{titlepage}

    \unitlength 1cm
    \begin{center}
    
    \vspace*{1cm}

    \includegraphics[scale=0.6]{figures/logo_ico.png}
    
    \vspace{2cm}
    
               {\Large Diplôme de Qualification en Physique Radiologique et Médicale\\}
               
    \vspace{2cm}           
    
    
    \rule{16cm}{0.7pt}
    
    \vspace{12pt}
               
               {\LARGE \bf Contrôle des distributions de dose\\}
               
    \vspace{12pt}
    \rule{16cm}{0.7pt}

    \vspace{2cm}

                {\large Fiche n°5}
    
    \vspace{1.5cm}

               {\Large\bf {Alexandre \textsc{Rintaud}}}
    
    \vspace{1.5cm}
    
    \end{center}
    
    Encadrantes :
    
    \small {
    \begin{tabular}{llr}\\
    \textbf{Sophie \textsc{Chiavassa}} et \textbf{Stéphanie \textsc{Josset}}  &  &  \\
      Physiciennes médicales, \textsc{Centre René Gauducheau ICO, Saint Herblain} &    &  \\
    
    \end{tabular}
    }

    \vspace{1.5cm}


    \begin{center}
    \textsc{Semestre 2 2023}
    \end{center}
    
\end{titlepage}
\let\cleardoublepage\clearpage


%%%%%%%%%%%%%%%%%%%%%%%%%%%%%%%%%%%%%%%%%%%%%%%%%%%%%%
%%%%%%%%%%%%%%%%%%%%%%% STYLE %%%%%%%%%%%%%%%%%%%%%%%%
%%%%%%%%%%%%%%%%%%%%%%%%%%%%%%%%%%%%%%%%%%%%%%%%%%%%%%

\onehalfspacing

%Style  du corps
\pagestyle{fancy}
	\renewcommand\headrulewidth{0.5pt}
	\renewcommand\footrulewidth{0.5pt}
	\fancyfoot[L]{\textsc{A. Rintaud}}
	\fancyfoot[C]{\textsc{ICO Nantes}}
	\fancyfoot[R]{\thepage}

\tableofcontents
\clearpage
\section{Introduction}

\section{Matériels et méthodes}

\subsection{Profils de dose}

Les profils de dose permettent d'évaluer la répartition de la dose suivant les axes perpendiculaires à l'axe du faisceau. Les profils sont composés de trois régions principales (voir figure \ref*{fig_regions_profil}) :

\begin{itemize}
  \item[$\bullet$] zone centrale
  \item[$\bullet$] pénombre
  \item[$\bullet$] dose hors champ
\end{itemize}

Pour évaluer les différents profils, plusieurs métriques sont à notre disposition :

\begin{itemize}
  \item[$\bullet$] L'homogénéité permet d'évaluer la planéité de la zone centrale du profil. Cette métrique se calcule comme suit :
  
  \begin{equation}
    H = \dfrac{D_{max} - D_{min}}{D_{max} + D_{min}}
    \label{eq_homogeneite}
  \end{equation}
  
  \item[$\bullet$] La symétrie est définie de la manière suivante :
  
  \begin{equation}
    S = \max\left(\dfrac{D(-x)}{D(+x)}; \dfrac{D(+x)}{D(-x)}\right)
    \label{eq_symetrie}
  \end{equation}
  
  \item[$\bullet$] La pénombre est la distance séparant le point à 20 \% de la dose maximale et celui à 80 \%. Il y a donc une valeur pour chacun des côtés du faisceau.  
\end{itemize}

\begin{figure}[h]
  \centering
  \includegraphics[scale=0.8]{figures/regions_profil.PNG}
  \caption{Différentes régions d'un profil de dose}
  \label{fig_regions_profil}
\end{figure}

\newpage
\section{Résultats}
\subsection{Rendements en profondeur}

\subsubsection{Influence de l'énergie}

\begin{figure}[h]
  \centering
  \includegraphics[scale=0.4]{../scripts/figures/rendements_energies.png}
  \caption{Inlfuence de l'énergie du faisceau d'électrons sur le rendement en profondeur}
  \label{fig_rdt_energie}
\end{figure}

\subsubsection{Influence de la taille de champ}

\begin{figure}[h]
  \centering
  \includegraphics[scale=0.4]{../scripts/figures/rendements_taille_champ.png}
  \caption{Inlfuence de la taille de champ du faisceau d'électrons sur le rendement en profondeur}
  \label{fig_rdt_champ}
\end{figure}

\subsubsection{Inlfuence de la DSP}

\begin{figure}[h!]
  \centering
  \includegraphics[scale=0.4]{../scripts/figures/rendements_DSP.png}
  \caption{Influence de la DSP sur le rendement en profondeur}
  \label{fig_rdt_DSP}
\end{figure}

\subsubsection{Influence du détecteur}

\begin{figure}[h!]
  \centering
  \includegraphics[scale=0.4]{../scripts/figures/rendements_detecteurs.png}
  \caption{Influence du détecteur sur le rendement en profondeur}
  \label{fig_rdt_detecteur}
\end{figure}

\subsection{Profils de dose}
\subsubsection{Inlfuence de l'énergie}

\begin{figure}[h]
  \centering
  \includegraphics[scale=0.4]{../scripts/figures/profils_energies.png}
  \caption{Influence de l'énergie du faisceau d'électrons sur le profil de dose}
  \label{fig_profil_energie}
\end{figure}

\subsubsection{Inlfuence de la taille de champ}

\begin{figure}[h]
  \centering
  \includegraphics[scale=0.4]{../scripts/figures/profils_taille_champ.png}
  \caption{Inluence de la taille de champ du faisceau d'électrons sur le profil de dose}
  \label{fig_profils_taille}
\end{figure}

\subsubsection{Inlfuence de la DSP}

\begin{figure}[h]
  \centering
  \includegraphics[scale=0.4]{../scripts/figures/profils_DSP.png}
  \caption{Influence de la DSP sur le profil de dose}
  \label{fig_profil_DSP}
\end{figure}

\subsubsection{Influence du détecteur}

\begin{figure}[h]
  \centering
  \includegraphics[scale=0.4]{../scripts/figures/profils_detecteurs.png}
  \caption{Influence du détecteur sur le profil de dose}
  \label{fig_profils_detecteur}
\end{figure}

\newpage
\subsection{Facteurs d'ouvertur du collimateur}

\begin{figure}[h]
  \centering
  \includegraphics[scale=0.4]{../scripts/figures/FOC.png}
  \caption{Facteurs d'ouverture du collimateur}
  \label{fig_foc}
\end{figure}

\clearpage
\bibliography{biblio}
\addcontentsline{toc}{section}{Références}
\bibliographystyle{plain}
\nocite{*}

\end{document}