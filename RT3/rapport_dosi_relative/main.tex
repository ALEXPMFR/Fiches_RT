\documentclass{article}
\usepackage[utf8]{inputenc}
\usepackage{amsmath}
\usepackage{amsfonts}
\usepackage{esint}
\usepackage{geometry}
\usepackage{color}
\usepackage{fancyhdr}
\usepackage{ctable}
\usepackage{fancybox}
\usepackage{tabularx}
\usepackage{array}
\usepackage{booktabs}
\usepackage[french]{babel}
\usepackage{dsfont}
\usepackage{setspace}
\usepackage[french]{minitoc}
\usepackage{multicol}
\usepackage{multirow}
\usepackage[hidelinks]{hyperref}
\usepackage{graphicx}
\usepackage[T1]{fontenc}
\usepackage{xcolor}
\usepackage{listings}

\geometry{top=2.5cm, bottom=2.5cm, left=3cm, right=3cm}

\addtocounter{tocdepth}{3}
\setcounter{secnumdepth}{3}


\definecolor{codegreen}{rgb}{0,0.6,0}
\definecolor{codegray}{rgb}{0.5,0.5,0.5}
\definecolor{codepurple}{rgb}{0.58,0,0.82}
\definecolor{backcolour}{rgb}{0.95,0.95,0.92}

\lstdefinestyle{mystyle}{
  backgroundcolor=\color{white}, commentstyle=\color{codegreen},
  keywordstyle=\color{magenta},
  numberstyle=\tiny\color{codegray},
  stringstyle=\color{codepurple},
  basicstyle=\ttfamily\footnotesize,
  breakatwhitespace=false,         
  breaklines=true,                 
  captionpos=b,                    
  keepspaces=true,                 
  numbers=left,                    
  numbersep=5pt,                  
  showspaces=false,                
  showstringspaces=false,
  showtabs=false,                  
  tabsize=2
}

\lstset{style=mystyle}

\begin{document}

%%%%%%%%%%%%%%%%%%%%%%%%%%%%%%%%%%%%%%%%%%%%%%%%%%%%%%
%%%%%%%%%%%%%%%%%%%% PRÉSENTATION %%%%%%%%%%%%%%%%%%%%
%%%%%%%%%%%%%%%%%%%%%%%%%%%%%%%%%%%%%%%%%%%%%%%%%%%%%%

\begin{titlepage}

    \unitlength 1cm
    \begin{center}
    
    \vspace*{1cm}

    \includegraphics[scale=0.6]{figures/logo_ico.png}
    
    \vspace{2cm}
    
               {\Large Diplôme de Qualification en Physique Radiologique et Médicale\\}
               
    \vspace{2cm}           
    
    
    \rule{16cm}{0.7pt}
    
    \vspace{12pt}
               
               {\LARGE \bf Contrôle des distributions de dose\\}
               
    \vspace{12pt}
    \rule{16cm}{0.7pt}

    \vspace{2cm}

                {\large Fiche n°5}
    
    \vspace{1.5cm}

               {\Large\bf {Alexandre \textsc{Rintaud}}}
    
    \vspace{1.5cm}
    
    \end{center}
    
    Encadrantes :
    
    \small {
    \begin{tabular}{llr}\\
    \textbf{Sophie \textsc{Chiavassa}} et \textbf{Stéphanie \textsc{Josset}}  &  &  \\
      Physiciennes médicales, \textsc{Centre René Gauducheau ICO, Saint Herblain} &    &  \\
    
    \end{tabular}
    }

    \vspace{1.5cm}


    \begin{center}
    \textsc{Semestre 2 2023}
    \end{center}
    
\end{titlepage}
\let\cleardoublepage\clearpage


%%%%%%%%%%%%%%%%%%%%%%%%%%%%%%%%%%%%%%%%%%%%%%%%%%%%%%
%%%%%%%%%%%%%%%%%%%%%%% STYLE %%%%%%%%%%%%%%%%%%%%%%%%
%%%%%%%%%%%%%%%%%%%%%%%%%%%%%%%%%%%%%%%%%%%%%%%%%%%%%%

\onehalfspacing

%Style  du corps
\pagestyle{fancy}
	\renewcommand\headrulewidth{0.5pt}
	\renewcommand\footrulewidth{0.5pt}
	\fancyfoot[L]{\textsc{A. Rintaud}}
	\fancyfoot[C]{\textsc{ICO Nantes}}
	\fancyfoot[R]{\thepage}

\tableofcontents
\clearpage
\section{Introduction}

La radiothérapie externe utilise, de manière prépondérante, les faisceux de photons de haute énergie afin de traiter des cellules cancéreuses tout en épargnant le plus possible les tissus sains. Dans cette optique, la connaissance précise des caractéristiques dosimétriques ainsi que les incertitudes associées de l'accélérateur utilisé sont nécessaires. 

Ce rapport traitera des faisceaux de photons utilisés en radiothérapie externe. era étudié l'influence de certains paramètres d'acquisition sur la dose relative. De plus, nous avons mesurée la dose absolue dans les conditions de référence en s'appuyant sur les protocoles internationaux fournis par l'Agence Internationale de l'Énerige Atomique (AIEA).

\section{Matériels et méthodes}
\subsection{Rendement en profondeur}


\subsection{Profils de dose}



\clearpage
\section{Résultats}
\subsection{Rendement en profondeur}
\subsubsection{Influence de l'énergie}

L'influence de l'énergie sur le rendement en profondeur pour un faisceau de photons est multiple. En effet, nous voyons tout d'abord sur la figure \ref*{fig_rdt_energie} que la zone de mise en équilibre électronique est plus grande pour un faisceau de 23 MV que pour un faisceau de 6 MV. Cela s'explique par le fait que plus les particules primaires sont énergétiques, plus l'énergie transférée aux électrons secondaires sera importante. Il faut donc une profondeur d'eau plus importante pour que la mise en équilibre électronique soit faite, donc la position du maximum de dépôt de dose est plus profonde. De plus, nous observons que le rendement est plus important, à une profondeur donnée, pour des photons de 23 MV par rapport aux photons de 6 MV. Cela est engendré par l'atténuation du milieu qui diminue pour une énergie qui augmente. Pour terminer, la dose à l'entée est plus faible pour le faisceau le plus énergétique car la probabilité d'interaction est plus faible à la surface d'entrée.

\begin{figure}[h]
  \centering
  \includegraphics[scale=0.4]{../scripts/figures/dose_relative/rdt_X6_X23.png}
  \caption{Influence de l'énergie du faisceau sur le rendement en profondeur}
  \label{fig_rdt_energie}
\end{figure}

\begin{table}[h]
  \centering
  \begin{tabular}{cccc}
    \toprule
    \textbf{Energie (MV)} & \textbf{R100 (cm)} & \textbf{R50 (cm)} & \textbf{Dose surface (\%)} \\
    \toprule
    6 & 1,23 & 15,07 & 55,96 \\
    23 & 3,16 & 21,24 & 33,7 \\
    \bottomrule
  \end{tabular}
  \caption{Résultats de l'influence de l'énergie sur les rendements en profondeur pour les faisceaux de photons du Clinac 2}
  \label{table_rdt_energie}
\end{table}

\subsubsection{Inlfuence de la DSP}

La figure \ref*{fig_rdt_dsp} nous montre l'influence de la DSP (distance source patient) sur le rendement en profondeur \footnote{La taille de champ à été adaptée pour qu'il y ait toujours un champ de 10x10 cm$^2$ à 10 cm de profondeur}. Nous pouvons voir que plus la DSP augmente, moins la dose relative est élevée.

La DSP n'a pas d'effet sur la zone de build-up ni sur la profondeur du maximum de dose.

\begin{figure}[h]
  \centering
  \includegraphics[scale=0.4]{../scripts/figures/dose_relative/DSP/Rendement_DSP.png}
  \caption{Influence de la DSP sur le rendement en profondeur}
  \label{fig_rdt_dsp}
\end{figure}

\begin{table}[h]
  \centering
  \begin{tabular}{cccc}
    \toprule
    \textbf{DSP (cm)} & \textbf{R100 (cm)} & \textbf{R50 (cm)} & \textbf{Dose surface (\%)} \\
    \toprule
    85 & 1,12 & 14,3 & 56,04 \\
    100 & 1,23 & 15,07 & 55,96 \\
    110 & 1,37 & 15,47 & 53.32 \\
    \bottomrule
  \end{tabular}
  \caption{Résultats de l'influence de la DSP sur les rendements en profondeur pour le faisceau de photons de 6 MV du Clinac 2}
  \label{table_rdt_dsp}
\end{table}

\subsubsection{Influence de la taille de champ}

La taille de champ influence d'une part la dose en profondeur, une fois que l'équilibre électronique est atteint. En effet, nous voyons que plus la taille de champ augmente,plus la dose en profondeur augmente. Ce phénomène est dû par l'augmentation du volume du milieu diffusant. D'autre part, la dose à l'entrée augmente avec la taille de champ. Cela s'explique par une quantité plus importante de rayonnement diffusé dans la tête de l'accélérateur (surface du cône égalisateur irradiée plus importante).

\begin{figure}[h]
  \centering
  \includegraphics[scale=0.4]{../scripts/figures/dose_relative/rendement_champs.png}
  \caption{Influence de la la taille de champ sur le rendement en profondeur}
  \label{fig_rdt_champ}
\end{figure}

\begin{table}[h]
  \centering
  \begin{tabular}{cccc}
    \toprule
    \textbf{Champ (cm}$\mathbf{^2}$\textbf{)} & \textbf{R100 (cm)} & \textbf{R50 (cm)} & \textbf{Dose surface (\%)} \\
    \toprule
    3x3 & 1,48 & 12,92 & 48,11 \\
    6x6 & 1,36 & 13,94 &  50,66 \\
    10x10 & 1,23 & 15,07 & 55,96 \\
    20x20 & 1,36 & 16,7 & 62,63 \\
    \bottomrule
  \end{tabular}
  \caption{Résultats de l'influence de la DSP sur les rendements en profondeur pour le faisceau de photons de 6 MV du Clinac 2}
  \label{table_rdt_dsp}
\end{table}

\subsubsection{Influence du détecteur}

Les mesures ont été faites avec plusieurs détecteurs qui sont :

\begin{itemize}
  \item[$\bullet$] CC13
  \item[$\bullet$] Diode SRS
  \item[$\bullet$] MicroDiamant
  \item[$\bullet$] Semiflex
  \item[$\bullet$] Pinpoint
\end{itemize}

La figure \ref*{fig_rdt_detecteurs} montre l'influence du détecteur choisi sur le rendement en profondeur. Nous pouvons voir que la diode mesure une pljus grande dose à la surface. Cela est dû au $Z$ effectif du détecteur qui est plus important que les autres détecteurs. Concernant les chambres d'ionisation Semiflex et CC13, leur volume sensible est très proche (0,013 cm$^3$ pour la CC13 et 0,125 cm$^3$ pour la Semiflex) ce qui explique leurs très fortes similarités. Nous observons que le signal de la Pinpoint et du MicroDiamant sont bruités. Leur volume sensible étant très petits (0,015 cm$^3$ et 0,004 mm$^3$ respectivement) la statistique de comptage est plus faible sur ces deux détecteurs puisque la vitesse d'acquisition est identique entre chacun des détecteurs.

\begin{figure}[h]
  \centering
  \includegraphics[scale=0.4]{../scripts/figures/dose_relative/test/Rendement.png}
  \caption{Influence du détecteur sur le profil d'un champ 10x10 cm$^2$}
  \label{fig_rdt_detecteurs}
\end{figure}

\begin{table}[h]
  \centering
  \begin{tabular}{cccc}
    \toprule
    \textbf{Détecteur} & \textbf{R100 (cm)} & \textbf{R50 (cm)} & \textbf{Dose surface (\%)} \\
    \toprule
    CC13 & 1,35 & 14,98 & 55,96 \\
    Diode & 0,7 & 14,55 & 90,72 \\
    MicroDiamant & 1,57 & 14,72 & 55,52 \\
    Pinpoint & 1,46 & 15,65 & 49,86 \\
    Semiflex & 1,29 & 14,96 & 55,68 \\
    \bottomrule
  \end{tabular}
  \caption{Résultats de l'influence du détecteur sur les rendements en profondeur pour le faisceau de photons de 6 MV du Clinac 2}
  \label{table_rdt_detecteurs}
\end{table}

\subsubsection{Inlfuence de la chambre de référence}

Pour s'affranchir de la flucutation du débit de dose du faisceau lors de la mesure de dose relative, une chambre dite de référence est placée dans l'air, dans un coin du champ d'irradiation pour ne pas perturber la mesure avec la chambre dans la cude à eau. Nous avons donc réalisé une mesure de rendement en profondeur pour observer ce rendement sans que la chambre de référence soit placée dans le champ. La figure \ref*{fig_sanss_chambre_ref} permet de voir l'importance de la présence de la chambre de référence. En effet, voyons que le signal est très bruité et ne permet donc pas d'analyser correctement les résultats.

\begin{figure}[h]
  \centering
  \includegraphics[scale=0.4]{../scripts/figures/dose_relative/chambre_ref.png}
  \caption{Influence de la présence de la chambre de référence}
  \label{fig_sanss_chambre_ref}
\end{figure}

\clearpage
\subsection{Profils de dose}
\subsubsection{Influence de l'énergie}

La pénombre augmente avec l'énergie, comme nous pouvons le voir sur la figure \ref*{fig_profils_energie} et le tableau \ref*{table_profils_energie}, car la transmission à travers les machoirs est plus importante et le cône égalisateur est différent entre le faisceau de 6 MV et celui de 23 MV.

\begin{figure}[h]
  \centering
  \includegraphics[scale=0.4]{../scripts/figures/dose_relative/profils/profils_energie.png}
  \caption{Influence de l'énergie du faisceau sur les profils}
  \label{fig_profils_energie}
\end{figure}

\begin{table}[h]
  \centering
  \begin{tabular}{>{\centering\arraybackslash}m{1.5cm}>{\centering\arraybackslash}m{2cm}>{\centering\arraybackslash}m{2cm}>{\centering\arraybackslash}m{2.5cm}>{\centering\arraybackslash}m{2.5cm}>{\centering\arraybackslash}m{3cm}}
    \toprule
    \textbf{Energie (MV)} & \textbf{Symétrie (\%)} & \textbf{Homogénéité (\%)} & \textbf{Centre du champ (cm)} & \textbf{Pénombre (G-D) (cm)} & \textbf{Taille de champ (cm)} \\
    \toprule
    6 & 101,38 & 2,55 & -0,01 & 0,76 - 0,755 & 11,13 \\
    23 & 101,15 & 2,5 & 0,03 & 0,87 - 0,86 & 11,15 \\
    \bottomrule
  \end{tabular}
  \caption{Influence de l'énergie sur les profils (résultats MyQA)}
  \label{table_profils_energie}
\end{table}

\subsubsection{Influence de la DSP}

Ayant adapté la taille de champ pour chacune des DSP, la figure \ref*{fig_profils_DSP} et le tableau \ref*{table_profils_dsp} montre très peu de différence. Cela s'explique par le fait que l'air est un milieu très peu diffusant et atténuant pour des photons de haute énergie. Nous pouvons donc considérer que la mesure de profils à profondeur constante en faisant varier la DSP n'influe pas sur la mesure.

\begin{figure}[h]
  \centering
  \includegraphics[scale=0.4]{../scripts/figures/dose_relative/DSP/Crossline_DSP.png}
  \caption{Influence de la DSP sur les profils de dose}
  \label{fig_profils_DSP}
\end{figure}

\begin{table}[h]
  \centering
  \begin{tabular}{>{\centering\arraybackslash}m{1.5cm}>{\centering\arraybackslash}m{2cm}>{\centering\arraybackslash}m{2cm}>{\centering\arraybackslash}m{2.5cm}>{\centering\arraybackslash}m{2.5cm}>{\centering\arraybackslash}m{3cm}}
    \toprule
    \textbf{DSP (cm)} & \textbf{Symétrie (\%)} & \textbf{Homogénéité (\%)} & \textbf{Centre du champ (cm)} & \textbf{Pénombre (G-D) (cm)} & \textbf{Taille de champ (cm)} \\
    \toprule
    85 & 101,28 & 2,35 & 0 & 0,67 - 0,67 & 11,08 \\
    100 & 100,43 & 2,37 & 0,01 & 0,68 - 0,68 & 11,07 \\
    110 & 101,81 & 2,39 & -0,04 & 0,69 - 0,69 & 11,12 \\
    \bottomrule
  \end{tabular}
  \caption{Influence de la DSP sur les profils (résultats MyQA)}
  \label{table_profils_dsp}
\end{table}

\subsubsection{Influence de la taille de champ}

Le tableau \ref*{table_profils_champ} montre que la pénombre augmente avec la taille de champ. Ce phénomène viens du fait que la surface du cône égalisateur est plus importante avec une grande taille de champ, ce qui augmente le rayonnement diffusé dans la tête de l'accélérateur. De plus, le fait d'augmenter la taille de champ augmente la surface apparente du collimateur, ce qui implique une augmentation du diffusé et donc de la pénombre. 

\begin{table}[h]
  \centering
  \begin{tabular}{>{\centering\arraybackslash}m{2cm}>{\centering\arraybackslash}m{1.5cm}>{\centering\arraybackslash}m{2cm}>{\centering\arraybackslash}m{2.3cm}>{\centering\arraybackslash}m{2.1cm}>{\centering\arraybackslash}m{2.3cm}}
    \toprule
    \textbf{Champ théorique (cm}$\mathbf{^2}$ \textbf{)} & \textbf{Symétrie (\%)} & \textbf{Homogénéité (\%)} & \textbf{Centre du champ (cm)} & \textbf{Pénombre (G-D) (cm)} & \textbf{Taille de champ (cm)} \\
    \toprule
    3x3 & 100,14 & 7,77 & 0 & 0,56 - 0,55 & 3,36 \\
    6x6 & 100,59 & 2,91 & 0,01 & 0,62 - 0,62 & 6,63 \\
    8x8 & 100,97 & 2,49 & -0,02 & 0,67 - 0,67 & 8,85 \\
    10x10 & 100,43 & 2,37 & 0,01 & 0,68 - 0,68 & 11,07 \\
    12x12 & 100,77 & 2,02 & -0,03 & 0,71 - 0,72 & 13,24 \\
    15x15 & 101,1 & 2,02 & 0,02 & 0,76 - 0,75 & 16,56 \\
    20x20 & 100,82 & 1,9 & -0,02 & 0,83 - 0,84 & 22,08 \\
    \bottomrule
  \end{tabular}
  \caption{Influence de la taille de champ sur les profils (résultats MyQA)}
  \label{table_profils_champ}
\end{table}

\begin{figure}[h]
  \centering
  \includegraphics[scale=0.4]{../scripts/figures/dose_relative/profils_champs/comp_champs.png}
  \caption{Influence de la taille de champ sur les profils de dose}
  \label{fig_champs_profils}
\end{figure}

\subsubsection{Influence du détecteur}

La figure \ref*{fig_profils_detecteurs} et le tableau \ref{table_profils_detecteurs} nous montre que la pénombre est très dépendante du détecteur. En effet, le choix du détecteur est crucial lorsqu'il y a de forts gradients de dose puisqu'en fonction du volume sensible du détecteur, les pentes des pénombres vont être modifiées. Nous pouvons voir qu'avec un détecteur possédant un volume sensible très petit (comme le MicroDiamant et la diode SRS) que la pénombre est petite. Si le volume sensible du détecteur est trop important, le signal reccueilli n'est pas représentatif de la réalité. Cependant, nous remarquons que pour la chambre d'ionisation Pinpoint, qui possède un volume sensible très petit de 0,015 cm$^3$, la dose est surrestimée au niveau des queues de distribution. Le bruit engendré par l'irradiation du manche de la chambre n'est plus négligeable par rapport au faible signal reccueilli par le petit volume sensible. Cela a pour effet d'augmenter la pénombre, ce qui est observable dans le tableau \ref{table_profils_detecteurs} (0,65 cm de pénombre pour la Pinpoint contre 0,39 en moyenne pour le MicroDiamant).

\begin{figure}[h]
  \centering
  \includegraphics[scale=0.4]{../scripts/figures/dose_relative/test/profils_10x10.png}
  \caption{Influence du détecteur sur les profils de dose}
  \label{fig_profils_detecteurs}
\end{figure}

\begin{table}[h]
  \centering
  \begin{tabular}{>{\centering\arraybackslash}m{1.7cm}>{\centering\arraybackslash}m{2cm}>{\centering\arraybackslash}m{2cm}>{\centering\arraybackslash}m{2.5cm}>{\centering\arraybackslash}m{2.2cm}>{\centering\arraybackslash}m{3cm}}
    \toprule
    \textbf{Détecteur} & \textbf{Symétrie (\%)} & \textbf{Homogénéité (\%)} & \textbf{Centre du champ (cm)} & \textbf{Pénombre (G-D) (cm)} & \textbf{Taille de champ (cm)} \\
    \toprule
    CC13 & 100,56 & 2,25 & -0,05 & 0,68 - 0,68 & 11,07 \\
    Semiflex & 100,9 & 2,53 & -0,04 & 0,67 - 0,67 & 11,1 \\
    Diode & 101,01 & 2,64 & -0,02 & 0,45 - 0,44 & 11,11 \\
    MicroDiamant & 101,72 & 2,91 & -0,04 & 0,38 - 0,4 & 11,04 \\
    Pinpoint & 101,99 & 2,68 & -0,02 & 0,65 - 0,65 & 11,07 \\
    \bottomrule
  \end{tabular}
  \caption{Influence du détecteur sur les profils (résultats MyQA)}
  \label{table_profils_detecteurs}
\end{table}

\subsubsection{Inlfuence du mode d'acquisition}

% \begin{figure}[h]
%   \centering
%   \includegraphics[scale=0.4]{../scripts/figures/dose_relative/}
% \end{figure}

\subsubsection{Influence de l'orientation du profil}

\begin{figure}[h]
  \centering
  \includegraphics[scale=0.4]{../scripts/figures/dose_relative/orientation_profil.png}
  \caption{Influence de l'orientation du profil}
  \label{fig_orientation_profil}
\end{figure}

\begin{table}[h]
  \centering
  \begin{tabular}{>{\centering\arraybackslash}m{1.7cm}>{\centering\arraybackslash}m{2cm}>{\centering\arraybackslash}m{2cm}>{\centering\arraybackslash}m{2.5cm}>{\centering\arraybackslash}m{2.5cm}>{\centering\arraybackslash}m{3cm}}
    \toprule
    \textbf{Orientation} & \textbf{Symétrie (\%)} & \textbf{Homogénéité (\%)} & \textbf{Centre du champ (cm)} & \textbf{Pénombre (G-D) (cm)} & \textbf{Taille de champ (cm)} \\
    \toprule
    Inline & 101,38 & 2,55 & -0,01 & 0,76 - 0,75 & 11,13 \\
    Crossline & 100,43 & 2,37 & 0,01 & 0,68 - 0,68 & 11,07 \\
    \bottomrule
  \end{tabular}
  \caption{Influence de l'orientation du profil (résultats MyQA)}
  \label{table_profils_dsp}
\end{table}

\clearpage
\subsection{Facteurs d'ouverture du collimateur (FOC)}

La figure \ref*{fig_foc} nous montre les FOC mesurés pour des champs allant de 3x3 cm$^2$ à 20x20 cm$^2$. Premièrement, lorsque l'énergie augmente, nous observons que les FOC croissent plus lentement. Ayant moins de rayonnements diffusés aux énergies élevées, le dépôt de dose est plus faible.

Lorsque une chambre d'ionisation avec un volume sensible élevé (chambre Farmer), les résultats sont quasiment identiques à l'exception du champ 3x3 cm$^2$. Cela est dû au trop grand volume sensible, la mesure est donc sous-estimée.

\begin{figure}[h]
  \centering
  \includegraphics[scale=0.4]{../scripts/figures/dose_relative/FOC.png}
  \caption{Facteurs d'ouverture du collimateur (FOC)}
  \label{fig_foc}
\end{figure}

\begin{table}[h]
  \centering
  \begin{tabular}{>{\centering\arraybackslash}m{1.5cm}>{\centering\arraybackslash}m{1.8cm}>{\centering\arraybackslash}m{1.8cm}>{\centering\arraybackslash}m{1.8cm}>{\centering\arraybackslash}m{1.8cm}>{\centering\arraybackslash}m{1.8cm}>{\centering\arraybackslash}m{1.8cm}}
  \toprule
  \textbf{Champ (cm}$\mathbf{^2}$ \textbf{)} &
  \textbf{Charge moyenne (nC)} &
  \textbf{FOC ref (\%)} &
  \textbf{FOC X23 (\%)} &
  \textbf{FOC Farmer (\%)} &
  \textbf{FOC DSP 80 (\%)} &
  \textbf{FOC DSP 120 (\%)} \\ \toprule
  3x3   & 4,846    & 82,44 & 84,13 & 81,12 & 82,54 & 83,35 \\
  6x6   & 5,391    & 91,71 & 94,38  & 91,74 & 91,97 & 91,88 \\
  8x8   & 5,654    & 96,18 & 97,74 & 96,38 & 96,47  & 96,39 \\
  10x10 & 5,878 & 100      & 100      & 100      & 100      & 100      \\
  12x12 & 6,051 & 102,93 & 101,83 & 102,99 & 102,79 & 102,89 \\
  15x15 & 6,261    & 106,51 & 103,79 & 106,45 & 106,27 & 106,59 \\
  20x20 & 6,521 & 110,94 & 106,06 & 110,55 & 110,45 & 111,08 \\ \bottomrule
  \end{tabular}
  \caption{Résultats des mesures des FOC}
  \label{table_resultats_foc}
\end{table}

% \clearpage
% \bibliography{biblio}
% \addcontentsline{toc}{section}{Références}
% \bibliographystyle{plain}
% \nocite{*}

\end{document}