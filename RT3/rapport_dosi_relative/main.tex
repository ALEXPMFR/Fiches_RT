\documentclass{article}
\usepackage[utf8]{inputenc}
\usepackage{amsmath}
\usepackage{amsfonts}
\usepackage{esint}
\usepackage{geometry}
\usepackage{color}
\usepackage{fancyhdr}
\usepackage{ctable}
\usepackage{fancybox}
\usepackage{tabularx}
\usepackage{array}
\usepackage{booktabs}
\usepackage{subcaption}
\usepackage[french]{babel}
\usepackage{dsfont}
\usepackage{setspace}
\usepackage[french]{minitoc}
\usepackage{multicol}
\usepackage{multirow}
\usepackage[hidelinks]{hyperref}
\usepackage{graphicx}
\usepackage[T1]{fontenc}
\usepackage{xcolor}
\usepackage{listings}

\geometry{top=2.3cm, bottom=2.3cm, left=2.5cm, right=2.5cm}

\addtocounter{tocdepth}{3}
\setcounter{secnumdepth}{3}


\definecolor{codegreen}{rgb}{0,0.6,0}
\definecolor{codegray}{rgb}{0.5,0.5,0.5}
\definecolor{codepurple}{rgb}{0.58,0,0.82}
\definecolor{backcolour}{rgb}{0.95,0.95,0.92}

\lstdefinestyle{mystyle}{
  backgroundcolor=\color{white}, commentstyle=\color{codegreen},
  keywordstyle=\color{magenta},
  numberstyle=\tiny\color{codegray},
  stringstyle=\color{codepurple},
  basicstyle=\ttfamily\footnotesize,
  breakatwhitespace=false,         
  breaklines=true,                 
  captionpos=b,                    
  keepspaces=true,                 
  numbers=left,                    
  numbersep=5pt,                  
  showspaces=false,                
  showstringspaces=false,
  showtabs=false,                  
  tabsize=2
}

\lstset{style=mystyle}

\begin{document}

%%%%%%%%%%%%%%%%%%%%%%%%%%%%%%%%%%%%%%%%%%%%%%%%%%%%%%
%%%%%%%%%%%%%%%%%%%% PRÉSENTATION %%%%%%%%%%%%%%%%%%%%
%%%%%%%%%%%%%%%%%%%%%%%%%%%%%%%%%%%%%%%%%%%%%%%%%%%%%%

\begin{titlepage}

    \unitlength 1cm
    \begin{center}
    
    \vspace*{1cm}

    \includegraphics[scale=0.6]{figures/logo_ico.png}
    
    \vspace{2cm}
    
               {\Large Diplôme de Qualification en Physique Radiologique et Médicale\\}
               
    \vspace{2cm}           
    
    
    \rule{16cm}{0.7pt}
    
    \vspace{12pt}
               
               {\LARGE \bf Contrôle des distributions de dose\\}
               
    \vspace{12pt}
    \rule{16cm}{0.7pt}

    \vspace{2cm}

                {\large Fiche n°5}
    
    \vspace{1.5cm}

               {\Large\bf {Alexandre \textsc{Rintaud}}}
    
    \vspace{1.5cm}
    
    \end{center}
    
    Encadrantes :
    
    \small {
    \begin{tabular}{llr}\\
    \textbf{Sophie \textsc{Chiavassa}} et \textbf{Stéphanie \textsc{Josset}}  &  &  \\
      Physiciennes médicales, \textsc{Centre René Gauducheau ICO, Saint Herblain} &    &  \\
    
    \end{tabular}
    }

    \vspace{1.5cm}


    \begin{center}
    \textsc{Semestre 2 2023}
    \end{center}
    
\end{titlepage}
\let\cleardoublepage\clearpage


%%%%%%%%%%%%%%%%%%%%%%%%%%%%%%%%%%%%%%%%%%%%%%%%%%%%%%
%%%%%%%%%%%%%%%%%%%%%%% STYLE %%%%%%%%%%%%%%%%%%%%%%%%
%%%%%%%%%%%%%%%%%%%%%%%%%%%%%%%%%%%%%%%%%%%%%%%%%%%%%%

\onehalfspacing

%Style  du corps
\pagestyle{fancy}
	\renewcommand\headrulewidth{0.5pt}
	\renewcommand\footrulewidth{0.5pt}
	\fancyfoot[L]{\textsc{A. Rintaud}}
	\fancyfoot[C]{\textsc{ICO Nantes}}
	\fancyfoot[R]{\thepage}

\tableofcontents
\clearpage
\section{Introduction}

La radiothérapie externe utilise, de manière prépondérante, les faisceaux de photons de haute énergie afin de traiter des cellules cancéreuses tout en épargnant le plus possible les tissus sains. Dans cette optique, la connaissance précise des caractéristiques dosimétriques ainsi que les incertitudes associées de l'accélérateur utilisé sont nécessaires. 

Ce rapport traitera des faisceaux de photons utilisés en radiothérapie externe. Sera étudié l'impact de certains paramètres d'acquisition sur la dose relative.

\section{Matériels et méthodes}

Les mesures de dose relative ont été réalisées à l'aide du logiciel MyQA du constructeur IBA, associé à la cuve à eau utilisée. Le tableau \ref*{table_mesures} regroupe l'ensemble des mesures réalisées au Clinac 2. Le tableau \ref*{table_mesures} énumère les différentes comparaisions faites lors des mesures. Le matériel utilisé est regroupé dans le tableau \ref*{table_matos}.

\begin{table}[h!]
  \centering
  \begin{tabular}{>{\centering\arraybackslash}m{1.5cm}>{\centering\arraybackslash}m{4cm}|>{\centering\arraybackslash}m{2.5cm}|>{\centering\arraybackslash}m{4cm}|}
    \cline{3-4}
    &                               & \textbf{Référence} & \textbf{Comparaison}                    \\ \hline
    \multicolumn{1}{|c|}{\multirow{5}{*}{\textbf{Rendements}}} & \textbf{Champ (cm}$\mathbf{^2}$\textbf{)} & 10x10 & 3x3, 6x6, 20x20                    \\
    \multicolumn{1}{|c|}{}                              & \textbf{DSP (cm)}             & 100                & 85, 110                                 \\
    \multicolumn{1}{|c|}{}                              & \textbf{Energie (MV)}         & 6                  & 23                                      \\
    \multicolumn{1}{|c|}{}                              & \textbf{Détecteur}            & CC13               & Pinpoint, MicroDiamant, Diode, Semiflex \\
    \multicolumn{1}{|c|}{}                              & \textbf{Chambre de référence} & Dans le champ      & Hors champ                              \\ \hline
    \multicolumn{1}{|c|}{\multirow{7}{*}{\textbf{Profils}}}    & \textbf{Champ (cm}$\mathbf{^2}$\textbf{)} & 10x10 & 3x3, 6x6, 8x8, 12x12, 15x15, 20x20 \\
    \multicolumn{1}{|c|}{}                              & \textbf{Profondeur (cm)}      & 10                 & 3, 20                                   \\
    \multicolumn{1}{|c|}{}                              & \textbf{Ouverture/Fermeture}  & Ouverture          & Fermeture                               \\
    \multicolumn{1}{|c|}{}                              & \textbf{Energie (MV)}         & 6                  & 23                                      \\
    \multicolumn{1}{|c|}{}                              & \textbf{DSP (cm)}             & 100                & 85, 110                                 \\
    \multicolumn{1}{|c|}{}                              & \textbf{Détecteur}            & CC13               & Pinpoint, MicroDiamant, Diode, Semiflex \\
    \multicolumn{1}{|c|}{}                              & \textbf{Orientation}          & Crosssline         & Inline                                  \\
    \multicolumn{1}{|c|}{}                              & \textbf{Mode d'acquisition}   & Continu & Step by step \\ \hline
    \multicolumn{1}{|c|}{\multirow{3}{*}{\textbf{FOC}}} & \textbf{Energie (MV)}         & 6                  & 23                                      \\
    \multicolumn{1}{|c|}{}                              & \textbf{Détecteur}            & CC13               & Farmer                                  \\
    \multicolumn{1}{|c|}{}                              & \textbf{DSP (cm)}             & 90                 & 80, 120                                 \\ \hline
  \end{tabular}
  \caption{Différentes mesures réalisées pour la dosimétrie relative}
  \label{table_mesures}
\end{table}

\begin{table}[h]
  \centering
  \begin{tabular}{>{\centering\arraybackslash}m{4cm}>{\centering\arraybackslash}m{3cm}>{\centering\arraybackslash}m{2.5cm}>{\centering\arraybackslash}m{2.5cm}>{\centering\arraybackslash}m{2cm}}
    \toprule
    \textbf{Matériel} & \textbf{Volume sensible (cm}$\mathbf{^3}$\textbf{)} & \textbf{Matériau} & \textbf{Constructeur} & \textbf{N}$\mathbf{^{\circ}}$\textbf{ de série}\\
    \toprule
    Chambre de référence CC13 & 0,13 & Air & IBA & 3922 \\
    Chambre de mesure CC13 & 0,13 & Air & IBA & 3923 \\
    Chambre Farmer 30013 & 0,6 & Air & PTW & 011924 \\
    Chambre Semiflex 31010 & 0,125 & Air & PTW & 008214 \\
    Chambre Pinpoint 31014 & 0,04 & Air & PTW & 00787 \\
    Diode SRS 60018 & 0,03 & Silicium & PTW & 000186 \\
    MicroDiamant 60019 & 0,004 $\times$ 10$^{-3}$ & Diamant & PTW & 122271 \\
    Electromètre Unidos & / & / & PTW & 20505 \\
    Cuve à eau Blue Phantom 2 & / & / & IBA & 8173 \\
    Clinac iX 2300 (Clinac 2) & / & / & Varian & H294581 \\
    \bottomrule
  \end{tabular}
  \caption{Récapitulatif du matériel utilisé lors des mesures}
  \label{table_matos}
\end{table}

Premièrement, les résultats concernant les rendements en profondeur seront présentés puis analysés. Il en sera de même pour les profils de dose et les facteurs d'ouverture du collimateur.

\newpage
\subsection{Rendement en profondeur}

Le rendement en profondeur (RDT) permet de connaître l'évolution de la dose dans le milieu de référence en fonction de la profondeur $z$ du point de mesure. Il est donné par la formule suivante :

\begin{equation}
    RDT(z, A, E) = \dfrac{D_z}{D_{max}}
    \label{eq_rdt}
\end{equation}

Avec :

\begin{itemize}
    \item[$\bullet$] $z$ la profondeur
    \item[$\bullet$] $A$ la taille de champ
    \item[$\bullet$] $DSP$ la distance source-patient
    \item[$\bullet$] $E$ l'énergie du faisceau
    \item[$\bullet$] $D_z$ la dose à la profondeur $z$
    \item[$\bullet$] $D_{max}$ la dose à la profondeur du maximum
\end{itemize}

\begin{figure}[h]
  \centering
  \includegraphics[scale=0.3]{figures/rdt_schema.png}
  \caption{Géométrie pour la mesure du rendement en profondeur}
  \label{fig_rdt}
\end{figure}

\subsection{Rapport tissu-fantôme}

Le rapport tiussu-fantôme (TPR) représente la dose en fonction de la profondeur. Le RTM est normalisé par rapport à la dose maximum. Contrairement au rendement en profondeur, la distance source-détecteur est fixe comme le montre la figure \ref*{fig_tpr}. Le TPR s'exprime à l'aide de la formule \ref{eq_tpr} :

\begin{equation}
    TPR(z, A, DSP, E) = \dfrac{D_z}{D_{ref}}
    \label{eq_tpr}
\end{equation}

Avec $D_{ref}$ la dose au point de référence.

\begin{figure}[h] 
  \centering
  \includegraphics[scale=0.3]{figures/tpr_schema.png}
  \caption{Géométrie pour la mesure du TPR}
  \label{fig_tpr}
\end{figure}

\subsection{Rapport tissu-maximum}

Le rapport tissu-maximum (RTM) est la normalisation par rapport à la dose à la profondeur $z_{max}$. Comme le TPR, la distance source-détecteur est constante (figure \ref*{fig_tpr}). 

Le RTM est un cas particulier du TPR puisqu'il est normalisé par rapport à la dose du maximum $z_{max}$. Il est défini par la fomule qui suit :

\begin{equation}
    RTM(z, A, E) = \dfrac{D_z}{D_{max}}
    \label{eq_rtm}
\end{equation}

Comme la distance entre la source et le détecteur est constante, la mesure du RTM est difficile à réaliser. La formule \ref{eq_calcul_rtm} permet de calculer le RTM à partir du rendement en profondeur :

\begin{equation}
    RTM(z, c, E) \approx RDT(z, c, DSP, E) \times \left(\dfrac{DSP + z}{DSP + z_{max}}\right)^2
    \label{eq_calcul_rtm}
\end{equation}

\subsection{Facteur d'ouverture du collimateur}

Le facteur d'ouverture du collimateur (FOC) est le rapport de la dose à une taille de champ $A$ quelconque et de la dose au champ de référence $A_{ref}$ :

\begin{equation}
    FOC(A, E, z) = \dfrac{D(A, E, z)}{D(c_{ref}, E, z)}
    \label{eq_foc}
\end{equation}

Avec :

\begin{itemize}
    \item[$\bullet$] $D(A, E, z)$ la dose à la taille de champ $A$, l'énergie $E$ et à la profondeur $z$
    \item[$\bullet$] $D(A_{ref}, E, z)$ la dose à la taille de champ de référence $A_{ref}$, à l'énergie $E$ et à la profondeur $z$
\end{itemize}

\begin{figure}[h]
  \centering
  \includegraphics[scale=0.6]{figures/foc_schema.png}
  \caption{Géométrie de mesure des FOC \cite{mayles2007handbook}}
  \label{fig_foc}
\end{figure}

\subsection{Profils de dose}

Les profils de dose permettent d'évaluer la répartition de la dose suivant les axes perpendiculaires à l'axe du faisceau. Les profils sont composés de trois régions principales (voir figure \ref*{fig_regions_profil}) :

\begin{itemize}
  \item[$\bullet$] zone centrale
  \item[$\bullet$] pénombre
  \item[$\bullet$] dose hors champ
\end{itemize}

Pour évaluer les différents profils, plusieurs métriques sont à notre disposition :

\begin{itemize}
  \item[$\bullet$] L'homogénéité permet d'évaluer la planéité de la zone centrale du profil. Cette métrique se calcule comme suit :
  
  \begin{equation}
    H = \dfrac{D_{max} - D_{min}}{D_{max} + D_{min}}
    \label{eq_homogeneite}
  \end{equation}
  
  \item[$\bullet$] La symétrie est définie de la manière suivante :
  
  \begin{equation}
    S = \max\left(\dfrac{D(-x)}{D(+x)}; \dfrac{D(+x)}{D(-x)}\right)
    \label{eq_symetrie}
  \end{equation}
  
  \item[$\bullet$] La pénombre est la distance séparant le point à 20 \% de la dose maximale et celui à 80 \%. Il y a donc une valeur pour chacun des côtés du faisceau.  
\end{itemize}

\begin{figure}[h]
  \centering
  \includegraphics[scale=0.8]{figures/regions_profil.PNG}
  \caption{Différentes régions d'un profil de dose}
  \label{fig_regions_profil}
\end{figure}

\section{Résultats}
\subsection{Rendement en profondeur}
\subsubsection{Influence de l'énergie}

L'influence de l'énergie sur le rendement en profondeur pour un faisceau de photons est multiple. En effet, nous voyons tout d'abord sur la figure \ref*{fig_rdt_energie} que la zone de mise en équilibre électronique est plus grande pour un faisceau de 23 MV que pour un faisceau de 6 MV. Cela s'explique par le fait que plus les particules primaires sont énergétiques, plus l'énergie transférée aux électrons secondaires sera importante. Il faut donc une profondeur d'eau plus importante pour que la mise en équilibre électronique soit faite, donc la position du maximum de dépôt de dose est plus profonde. De plus, nous observons que le rendement est plus important, à une profondeur donnée, pour des photons de 23 MV par rapport aux photons de 6 MV pour une profondeur supérieure au $z_{max}$. Cela est engendré par l'atténuation du milieu qui diminue pour une énergie qui augmente. Pour terminer, la dose à l'entrée est plus faible pour le faisceau le plus énergétique car la probabilité d'interaction est plus faible à la surface d'entrée.

\begin{figure}[h]
  \centering
  \includegraphics[scale=0.45]{../scripts/figures/dose_relative/rdt_X6_X23.png}
  \caption{Influence de l'énergie du faisceau sur le rendement en profondeur}
  \label{fig_rdt_energie}
\end{figure}

\begin{table}[h]
  \centering
  \begin{tabular}{cccc}
    \toprule
    \textbf{Energie (MV)} & \textbf{R100 (cm)} & \textbf{R50 (cm)} & \textbf{Dose surface (\%)} \\
    \toprule
    6 & 1,23 & 15,07 & 55,96 \\
    23 & 3,16 & 21,24 & 33,7 \\
    \bottomrule
  \end{tabular}
  \caption{Résultats de l'influence de l'énergie sur les rendements en profondeur pour les faisceaux de photons du Clinac 2}
  \label{table_rdt_energie}
\end{table}

\subsubsection{Influence de la DSP}

La figure \ref*{fig_rdt_dsp} nous montre l'influence de la DSP (distance source patient) sur le rendement en profondeur \footnote{La taille de champ à été adaptée pour qu'il y ait toujours un champ de 10x10 cm$^2$ à 10 cm de profondeur}. Nous pouvons voir que plus la DSP augmente, plus la dose relative est élevée.

La DSP agît surtout en profondeur. La pente du rendement en profondeur est plus faible pour l'acquisition à 110 cm de DSP. Cela peut s'expliquer par l'effet de la loi d'inverse carrée de la distance, qui est plus important à distance plus faible, d'où une dose plus faible pour une DSP plus petite pour une même profondeur.

\begin{figure}[h]
  \centering
  \includegraphics[scale=0.44]{../scripts/figures/dose_relative/DSP/Rendement_DSP.png}
  \caption{Influence de la DSP sur le rendement en profondeur}
  \label{fig_rdt_dsp}
\end{figure}

\begin{table}[h]
  \centering
  \begin{tabular}{cccc}
    \toprule
    \textbf{DSP (cm)} & \textbf{R100 (cm)} & \textbf{R50 (cm)} & \textbf{Dose surface (\%)} \\
    \toprule
    85 & 1,12 & 14,3 & 56,04 \\
    100 & 1,23 & 15,07 & 55,96 \\
    110 & 1,37 & 15,47 & 53.32 \\
    \bottomrule
  \end{tabular}
  \caption{Résultats de l'influence de la DSP sur les rendements en profondeur pour le faisceau de photons de 6 MV du Clinac 2}
  \label{table_rdt_dsp}
\end{table}

\subsubsection{Influence de la taille de champ}

La taille de champ influence d'une part la dose en profondeur, une fois que l'équilibre électronique est atteint. En effet, nous voyons que plus la taille de champ augmente, plus la dose en profondeur augmente. Ce phénomène est expliqué par l'augmentation du volume du milieu diffusant. D'autre part, la dose à l'entrée augmente avec la taille de champ. Cela s'explique par une quantité plus importante de rayonnement diffusé dans la tête de l'accélérateur (surface du cône égalisateur irradiée plus importante et surface apparente des machoires plus grande).

\begin{figure}[h!]
  \centering
  \includegraphics[scale=0.45]{../scripts/figures/dose_relative/rendement_champs.png}
  \caption{Influence de la la taille de champ sur le rendement en profondeur}
  \label{fig_rdt_champ}
\end{figure}

\begin{table}[h]
  \centering
  \begin{tabular}{cccc}
    \toprule
    \textbf{Champ (cm}$\mathbf{^2}$\textbf{)} & \textbf{R100 (cm)} & \textbf{R50 (cm)} & \textbf{Dose surface (\%)} \\
    \toprule
    3x3 & 1,48 & 12,92 & 48,11 \\
    6x6 & 1,36 & 13,94 &  50,66 \\
    10x10 & 1,23 & 15,07 & 55,96 \\
    20x20 & 1,36 & 16,7 & 62,63 \\
    \bottomrule
  \end{tabular}
  \caption{Résultats de l'influence de la DSP sur les rendements en profondeur pour le faisceau de photons de 6 MV du Clinac 2}
  \label{table_rdt_dsp}
\end{table}

\newpage
\subsubsection{Influence du détecteur}

Les mesures ont été faites avec plusieurs détecteurs qui sont :

\begin{itemize}
  \item[$\bullet$] CC13
  \item[$\bullet$] Diode SRS
  \item[$\bullet$] MicroDiamant
  \item[$\bullet$] Semiflex
  \item[$\bullet$] Pinpoint
\end{itemize}

La figure \ref*{fig_rdt_detecteurs} montre l'influence du détecteur choisi sur le rendement en profondeur. Nous pouvons voir que la diode mesure une plus grande dose à la surface. Cela est dû au $Z$ effectif du détecteur qui est plus important que les autres détecteurs ($Z$ = 14). Concernant les chambres d'ionisation Semiflex et CC13, leur volume sensible est très proche (0,013 cm$^3$ pour la CC13 et 0,125 cm$^3$ pour la Semiflex) ce qui explique leurs très fortes similarités. Nous observons que le signal de la Pinpoint et du MicroDiamant sont bruités. Leur volume sensible étant très petits (0,015 cm$^3$ et 0,004 mm$^3$ respectivement) la statistique de comptage est plus faible sur ces deux détecteurs puisque la vitesse d'acquisition est identique entre chacun des détecteurs.

\begin{figure}[h]
  \centering
  \includegraphics[scale=0.45]{../scripts/figures/dose_relative/test/Rendement.png}
  \caption{Influence du détecteur sur le profil d'un champ 10x10 cm$^2$}
  \label{fig_rdt_detecteurs}
\end{figure}

\begin{table}[h]
  \centering
  \begin{tabular}{cccc}
    \toprule
    \textbf{Détecteur} & \textbf{R100 (cm)} & \textbf{R50 (cm)} & \textbf{Dose surface (\%)} \\
    \toprule
    CC13 & 1,35 & 14,98 & 55,96 \\
    Diode & 0,7 & 14,55 & 90,72 \\
    MicroDiamant & 1,57 & 14,72 & 55,52 \\
    Pinpoint & 1,46 & 15,65 & 49,86 \\
    Semiflex & 1,29 & 14,96 & 55,68 \\
    \bottomrule
  \end{tabular}
  \caption{Résultats de l'influence du détecteur sur les rendements en profondeur pour le faisceau de photons de 6 MV du Clinac 2}
  \label{table_rdt_detecteurs}
\end{table}

\newpage
\subsubsection{Influence de la chambre de référence}

\begin{figure}[h!]
  \centering
  \includegraphics[scale=0.4]{../scripts/figures/dose_relative/chambre_ref.png}
  \caption{Influence de la présence de la chambre de référence}
  \label{fig_sanss_chambre_ref}
\end{figure}

Pour s'affranchir de la fluctuation du débit de dose du faisceau lors de la mesure de dose relative, une chambre dite de référence est placée dans l'air, dans un coin du champ d'irradiation pour ne pas perturber la mesure avec la chambre dans la cuve à eau. Nous avons donc réalisé une mesure de rendement en profondeur pour observer ce rendement sans que la chambre de référence soit placée dans le champ. La figure \ref*{fig_sanss_chambre_ref} permet de voir l'importance de la présence de la chambre de référence. En effet, nous voyons que le signal est très bruité et ne permet donc pas d'analyser correctement les résultats.

\subsection{Profils de dose}
\subsubsection{Influence de l'énergie}

La pénombre augmente avec l'énergie, comme nous pouvons le voir sur la figure \ref*{fig_profils_energie} et le tableau \ref*{table_profils_energie}, car la transmission à travers les machoires est plus importante et le cône égalisateur est différent entre le faisceau de 6 MV et celui de 23 MV.

\begin{figure}[h]
  \centering
  \includegraphics[scale=0.45]{../scripts/figures/dose_relative/profils/profils_energie.png}
  \caption{Influence de l'énergie du faisceau sur les profils}
  \label{fig_profils_energie}
\end{figure}

\begin{table}[h]
  \centering
  \begin{tabular}{>{\centering\arraybackslash}m{1.5cm}>{\centering\arraybackslash}m{2cm}>{\centering\arraybackslash}m{2cm}>{\centering\arraybackslash}m{2.5cm}>{\centering\arraybackslash}m{2.3cm}>{\centering\arraybackslash}m{2.5cm}}
    \toprule
    \textbf{Energie (MV)} & \textbf{Symétrie (\%)} & \textbf{Homogénéité (\%)} & \textbf{Centre du champ (cm)} & \textbf{Pénombre (G-D) (cm)} & \textbf{Taille de champ (cm)} \\
    \toprule
    6 & 101,38 & 2,55 & -0,01 & 0,76 - 0,755 & 11,13 \\
    23 & 101,15 & 2,5 & 0,03 & 0,87 - 0,86 & 11,15 \\
    \bottomrule
  \end{tabular}
  \caption{Influence de l'énergie sur les profils (résultats MyQA)}
  \label{table_profils_energie}
\end{table}

\newpage
\subsubsection{Influence de la DSP}

Ayant adapté la taille de champ pour chacune des DSP, la figure \ref*{fig_profils_DSP} et le tableau \ref*{table_profils_dsp} montrent très peu de différences. Cela s'explique par le fait que l'air est un milieu très peu diffusant et atténuant pour des faisceaux de photons de haute énergie. Nous pouvons donc considérer que la mesure de profils à profondeur constante en faisant varier la DSP n'influe que très peu sur la mesure.

\begin{figure}[h]
  \centering
  \includegraphics[scale=0.448]{../scripts/figures/dose_relative/DSP/Crossline_DSP.png}
  \caption{Influence de la DSP sur les profils de dose}
  \label{fig_profils_DSP}
\end{figure}



\begin{table}[h]
  \centering
  \begin{tabular}{>{\centering\arraybackslash}m{1.5cm}>{\centering\arraybackslash}m{2cm}>{\centering\arraybackslash}m{2cm}>{\centering\arraybackslash}m{2.5cm}>{\centering\arraybackslash}m{2.3cm}>{\centering\arraybackslash}m{2.5cm}}
    \toprule
    \textbf{DSP (cm)} & \textbf{Symétrie (\%)} & \textbf{Homogénéité (\%)} & \textbf{Centre du champ (cm)} & \textbf{Pénombre (G-D) (cm)} & \textbf{Taille de champ (cm)} \\
    \toprule
    85 & 101,28 & 2,35 & 0 & 0,67 - 0,67 & 11,08 \\
    100 & 100,43 & 2,37 & 0,01 & 0,68 - 0,68 & 11,07 \\
    110 & 101,81 & 2,39 & -0,04 & 0,69 - 0,69 & 11,12 \\
    \bottomrule
  \end{tabular}
  \caption{Influence de la DSP sur les profils (résultats MyQA)}
  \label{table_profils_dsp}
\end{table}

\subsubsection{Influence de la taille de champ}

Pour observer l'influence de la taille de champ sur les profils de dose, nous avons réalisé des acquisitions avec des champs allant de 3$\times$3 cm$^2$ à 20$\times$20 cm$^2$. Le tableau \ref*{table_profils_champ} montre que la pénombre augmente avec la taille de champ. Ce phénomène vient du fait que la surface apparente du cône égalisateur est plus élevée à grande ouverture par rapport à un petit champ, ce qui augmente le rayonnement diffusé dans la tête de l'accélérateur. De plus, le fait d'augmenter la taille de champ augmente la surface apparente du collimateur, ce qui implique une augmentation du diffusé et donc de la pénombre. Nous pouvons voir également sur le tableau \ref*{table_profils_champ} que l'homogénéité augmente lorsque la taille de champ diminue. Cette grandeur est mesurée sur la partie plane du profil. Or, pour des petits champs, le profil n'est plus vraiment plat mais se rapproche de la forme du gaussienne (voir figure \ref*{fig_champs_profils}), il est donc comprehensible que le résultat de l'homogénéité soit affecté pour de telles tailles de champ.

\begin{figure}[h]
  \centering
  \includegraphics[scale=0.4]{../scripts/figures/dose_relative/profils_champs/comp_champs.png}
  \caption{Influence de la taille de champ sur les profils de dose}
  \label{fig_champs_profils}
\end{figure}

\begin{table}[h]
  \centering
  \begin{tabular}{>{\centering\arraybackslash}m{2cm}>{\centering\arraybackslash}m{1.5cm}>{\centering\arraybackslash}m{2cm}>{\centering\arraybackslash}m{2.3cm}>{\centering\arraybackslash}m{2.1cm}>{\centering\arraybackslash}m{2.3cm}}
    \toprule
    \textbf{Champ théorique (cm}$\mathbf{^2}$ \textbf{)} & \textbf{Symétrie (\%)} & \textbf{Homogénéité (\%)} & \textbf{Centre du champ (cm)} & \textbf{Pénombre (G-D) (cm)} & \textbf{Taille de champ (cm)} \\
    \toprule
    3x3 & 100,14 & 7,77 & 0 & 0,56 - 0,55 & 3,36 \\
    6x6 & 100,59 & 2,91 & 0,01 & 0,62 - 0,62 & 6,63 \\
    8x8 & 100,97 & 2,49 & -0,02 & 0,67 - 0,67 & 8,85 \\
    10x10 & 100,43 & 2,37 & 0,01 & 0,68 - 0,68 & 11,07 \\
    12x12 & 100,77 & 2,02 & -0,03 & 0,71 - 0,72 & 13,24 \\
    15x15 & 101,1 & 2,02 & 0,02 & 0,76 - 0,75 & 16,56 \\
    20x20 & 100,82 & 1,9 & -0,02 & 0,83 - 0,84 & 22,08 \\
    \bottomrule
  \end{tabular}
  \caption{Influence de la taille de champ sur les profils (résultats MyQA)}
  \label{table_profils_champ}
\end{table}

\newpage
\subsubsection{Influence du détecteur}

La figure \ref*{fig_profils_detecteurs} et le tableau \ref{table_profils_detecteurs} nous montre que la pénombre est très dépendante du détecteur. En effet, le choix du détecteur est crucial lorsqu'il y a de forts gradients de dose puisqu'en fonction du volume sensible du détecteur, les pentes des pénombres vont être modifiées. Nous pouvons voir qu'avec un détecteur possédant un volume sensible très petit (comme le MicroDiamant et la diode SRS) que la pénombre est petite. Si le volume sensible du détecteur est trop important, le signal reccueilli n'est pas représentatif de la réalité. Cependant, nous remarquons que pour la chambre d'ionisation Pinpoint, qui possède un volume sensible très petit de 0,015 cm$^3$, la dose est surrestimée au niveau des queues de distribution. Le bruit engendré par l'irradiation du manche de la chambre n'est plus négligeable par rapport au faible signal reccueilli par le petit volume sensible. Cela a pour effet d'augmenter la pénombre, ce qui est observable dans le tableau \ref{table_profils_detecteurs} (0,65 cm de pénombre pour la Pinpoint contre 0,39 en moyenne pour le MicroDiamant). La chambre Pinpoint étant composée d'un volume d'air, le pouvoir d'arrêt est plus faible que pour le volume sensible de la diode SRS et du détecteru MicroDiamant.

\begin{figure}[h]
  \centering
  \includegraphics[scale=0.45]{../scripts/figures/dose_relative/test/profils_10x10.png}
  \caption{Influence du détecteur sur les profils de dose}
  \label{fig_profils_detecteurs}
\end{figure}

\begin{table}[h]
  \centering
  \begin{tabular}{>{\centering\arraybackslash}m{1.7cm}>{\centering\arraybackslash}m{2cm}>{\centering\arraybackslash}m{2cm}>{\centering\arraybackslash}m{2.5cm}>{\centering\arraybackslash}m{2.2cm}>{\centering\arraybackslash}m{3cm}}
    \toprule
    \textbf{Détecteur} & \textbf{Symétrie (\%)} & \textbf{Homogénéité (\%)} & \textbf{Centre du champ (cm)} & \textbf{Pénombre G-D (cm)} & \textbf{Taille de champ (cm)} \\
    \toprule
    CC13 & 100,56 & 2,25 & -0,05 & 0,68 - 0,68 & 11,07 \\
    Semiflex & 100,9 & 2,53 & -0,04 & 0,67 - 0,67 & 11,1 \\
    Diode & 101,01 & 2,64 & -0,02 & 0,45 - 0,44 & 11,11 \\
    MicroDiamant & 101,72 & 2,91 & -0,04 & 0,38 - 0,4 & 11,04 \\
    Pinpoint & 101,99 & 2,68 & -0,02 & 0,65 - 0,65 & 11,07 \\
    \bottomrule
  \end{tabular}
  \caption{Influence du détecteur sur les profils (résultats MyQA)}
  \label{table_profils_detecteurs}
\end{table}

\newpage
\subsubsection{Influence de l'orientation du profil}

Lors de l'acquisition des profils, nous pouvons choisir l'axe selon lequel le profil sera enregistré. Les résultats des deux profils sont donnés dans la tableau \ref*{table_profils_orientation} et sur la figure \ref*{fig_orientation_profil}. Les résultats montrent que la pénombre est modifiée entre ces deux acquistions. En effet, la pénombre est plus importante en \textit{inline} (tête-pieds) par rapport à l'orientation \textit{crossline} (droite-gauche). Ce sont les machoîres qui sont responsables de cette différence car les deux paires de machoîres ne sont pas sur le même plan. Celles qui définissent le champ en \textit{crossline} sont en-dessous de celles qui le définissent en \textit{inline}.

\begin{figure}[h]
  \centering
  \includegraphics[scale=0.44]{../scripts/figures/dose_relative/orientation_profil.png}
  \caption{Influence de l'orientation du profil}
  \label{fig_orientation_profil}
\end{figure}

\begin{table}[h]
  \centering
  \begin{tabular}{>{\centering\arraybackslash}m{1.7cm}>{\centering\arraybackslash}m{2cm}>{\centering\arraybackslash}m{2cm}>{\centering\arraybackslash}m{2.5cm}>{\centering\arraybackslash}m{2.5cm}>{\centering\arraybackslash}m{3cm}}
    \toprule
    \textbf{Orientation} & \textbf{Symétrie (\%)} & \textbf{Homogénéité (\%)} & \textbf{Centre du champ (cm)} & \textbf{Pénombre (G-D) (cm)} & \textbf{Taille de champ (cm)} \\
    \toprule
    Inline & 101,38 & 2,55 & -0,01 & 0,76 - 0,75 & 11,13 \\
    Crossline & 100,43 & 2,37 & 0,01 & 0,68 - 0,68 & 11,07 \\
    \bottomrule
  \end{tabular}
  \caption{Influence de l'orientation du profil (résultats MyQA)}
  \label{table_profils_orientation}
\end{table}

\newpage
\subsubsection{Influence du mode d'acquisition}

Pour observer l'impact du mode d'acquisition sur les profils de dose, nous avons réalisé plusieurs mesures, comme nous plouvons le voir dans le tableau \ref*{table_ss}.

% \begin{itemize}
%   \item[$\bullet$] Continu avec une vitesse de déplacement de chambre de 0,3 cm/s
%   \item[$\bullet$] Continu avec une vitesse de déplacement de chambre de 2,5 cm/s
%   \item[$\bullet$] Pas à pas avec une mesure tous les 0,1 cm sur les côtés et une mesure centrale tous les 0,5 cm avec un temps d'intéragtion d'une seconde pour chaque point
%   \item[$\bullet$] Pas à pas avec une mesure tous les 0,1 cm sur les côtés et une mesure centrale tous les 0,5 cm avec un temps d'intéragtion d'une demi seconde pour chaque point
%   \item[$\bullet$] Pas à pas avec une mesure tous les 0,1 cm sur les côtés et une mesure centrale tous les 0,5 cm avec un temps d'intéragtion de trois secondes pour chaque point
%   \item[$\bullet$] Pas à pas avec une mesure tous les 0,5 cm sur les côtés et une mesure centrale tous les centimètres avec un temps d'intéragtion d'une seconde pour chaque point
%   \item[$\bullet$] Pas à pas avec une mesure tous les 0,05 cm sur les côtés et une mesure centrale tous les 0,2 cm avec un temps d'intéragtion d'une seconde pour chaque point
% \end{itemize}

\begin{figure}[h]
  \centering
  \begin{subfigure}{\textwidth}
    \centering
    \includegraphics[scale=0.35]{../scripts/figures/dose_relative/step_by_step/step_by_step_1.png}
  \end{subfigure}
  \vspace{0.5cm}
  \begin{subfigure}{\textwidth}
    \centering
    \includegraphics[scale=0.36]{../scripts/figures/dose_relative/step_by_step/step_by_step_2.png}
  \end{subfigure}
  \caption{Influence du mode d'acquisition sur les profils de dose}
  \label{fig_mode_acquisition}
\end{figure}

\begin{table}[h]
  \centering
  \begin{tabular}{>{\centering\arraybackslash}m{2cm}>{\centering\arraybackslash}m{1cm}>{\centering\arraybackslash}m{2cm}>{\centering\arraybackslash}m{1cm}>{\centering\arraybackslash}m{1.5cm}>{\centering\arraybackslash}m{0.7cm}>{\centering\arraybackslash}m{0.7cm}>{\centering\arraybackslash}m{1.7cm}>{\centering\arraybackslash}m{1.7cm}}
    \toprule
    \textbf{Mode acquisition} & \textbf{Vitesse (cm/s)} & \textbf{Pas G-D (cm)} & \textbf{Pas centre (cm)} & \textbf{Temps d'intégration (s)} & \textbf{S (\%)} & \textbf{H (\%)} & \textbf{Taille de champ (cm)} & \textbf{Pénombre G-D (cm)}\\
    \toprule
    Continu & 0,3 & / & / & / & 2,36 & 100,86 & 11,08 & 0,67 - 0,68\\
    Continu & 2,5 & / & / & / & 2,51 & 100,83 & 11,08 & 0,69 - 0,69\\
    Steb by Step & / & 0,1 & 0,5 & 1 & 2,23 & 100,67 & 11,08 & 0,66 - 0,67\\
    Step by Step & / & 0,1 & 0,5 & 0,5 & 2,26 & 100,53 & 11,08 & 0,67 - 0,67\\
    Step by step & / & 0,1 & 0,5 & 3 & 2,27 & 100,66 & 11,08 & 0,68 - 0,67\\
    Step by Step & / & 0,5 & 1 & 1 & 2,41 & 100,78 & 11,11 & 0,80 - 0,85\\
    Step by step & / & 0,05 & 0,2 & 1 & 2,26 & 100,61 & 11,08 & 0,66 - 0,67 \\
    \bottomrule
  \end{tabular}
  \caption{Inlfuence du mode d'acquisition sur les profils (résultats MyQA)}
  \label{table_ss}
\end{table}

Premièrement, concernant le mode continu, nous voyons sur le tableau \ref*{table_ss} que la différence de pénombre entre les vitesses lentes et rapides est relativement faible (0,015 cm de différence en moyenne). Choisir une vitesse intermédiaire est satisfaisant pour des mesures rapides et précises. De plus, nous pouvons voir que le mode pas à pas avec une mesure tous les 0,05 cm sur les côtés du profils et tous les 0,2 cm au centre permet d'obtenir la plus petite pénombre mais n'est pas significatif pour les modes intermédiaires. Cependant, choisir un pas d'intégration trop grand n'est pas non plus optimal car la pénombre mesurée est plus grande mais également non symétrique car le logiciel doit interpoler entre les points pour obtenir les abscisses des points à 20\% et à 80\% de la dose, ce qui fausse la mesure de la pénombre.

\subsection{Perturbations liées à la chambre de référence}

La figure \ref*{fig_perturbations_chambre_ref_profils} et le tableau \ref*{table_resultats_perturb_chambre_ref} nous montrent les résultats de l'influence de la position de la chambre de référence dans le champ d'irradiation. Comme nous pouvons le voir sur le graphe et le tableau, le fait que la chambre de référence soit placée dans le coin ou presque au centre du champ ne change rien sur les résultats de l'acquisition puisque les résultats fournis par le tableau \ref*{table_resultats_perturb_chambre_ref} nous montrent que la symétrie, l'homogénéité et la pénombre sont extêmement proches. Nous pouvons conclure que la position de la chambre de référence perturbe très peu la signal reccueilli par la chambre de mesure.

\begin{figure}[h!]
  \centering
  \includegraphics[scale=0.41]{../scripts/figures/dose_relative/perturbations_chambre/perturbations_X6.png}
  \caption{Influence de la localisation de la chambre de référence sur les profils de dose}
  \label{fig_perturbations_chambre_ref_profils}
\end{figure}

\begin{table}[h]
  \centering
  \begin{tabular}{>{\centering\arraybackslash}m{2.5cm}>{\centering\arraybackslash}m{1.1cm}>{\centering\arraybackslash}m{1.1cm}>{\centering\arraybackslash}m{2.5cm}>{\centering\arraybackslash}m{2cm}>{\centering\arraybackslash}m{2.5cm}}
    \toprule
    \textbf{Position chambre ref} & \textbf{S (\%)} & \textbf{H (\%)} & \textbf{Centre du champ (cm)} & \textbf{Pénombre G-D (cm)} & \textbf{Taille de champ (cm)} \\
    \toprule
    Coin & 100,43 & 2,37 & 0,01 & 0,68 - 0,68 & 11,07 \\
    Centre & 100,69 & 2,29 & 0 & 0,68 - 0,68 & 11,12 \\
    \bottomrule
  \end{tabular}
  \caption{Influence de la position de la chambre de référence sur les profils de dose (résultats MyQA)}
  \label{table_resultats_perturb_chambre_ref}
\end{table}

\subsection{Facteurs d'ouverture du collimateur (FOC)}

Le tableau \ref*{table_resultats_foc} et la figure \ref*{fig_foc} nous montre les FOC mesurés pour des champs allant de 3x3 cm$^2$ à 20x20 cm$^2$. Premièrement, nous voyons que pour des champs inférieurs à 10$\times$10 cm$^2$ les FOC sont plus importants pour le faisceau de 6 MV que pour celui de 23 MV. Cela peu s'expliquer par le fait qu'à plus faible énergie la quantité d'électrons rétrodiffusés vers les chambres monitrices est plus importante (pour de petits champs), la dose mesurée par les chambres monitrices sera plus importante que la réalité et la coupure du faisceau sera donc prématurée. A contrario, pour une grande taille de champ, ce phénomène devient de plus en plus négligeable et la quantité de photons diffusés dans le fantôme d'eau à plus faible énergie prend le dessus. Ceci explique pourquoi les FOC sont plus faible pour une énergie de 23 MV plutôt qu'à 6 MV pour les grandes tailles de champ (et inversement).

Lorsqu'on utilise une chambre d'ionisation avec un volume sensible élevé (chambre de type Farmer), les résultats des FOC sont quasiment identiques par rapport à la chambrer CC13, à l'exception du champ 3x3 cm$^2$. Cela est dû au trop grand volume sensible du détecteur par rapport aux dimensions du champ, la mesure est donc sous-estimée avec la chambre Farmer.

Enfin,  nous observons que le rôle de la DSP n'est pas significatif. En effet, la taille de champ ayant été adaptée, nous pouvons considérer que l'interaction du faisceau avec l'air (entre la source et le fantôme) est négligeable.

\begin{figure}[h]
  \centering
  \includegraphics[scale=0.45]{../scripts/figures/dose_relative/FOC.png}
  \caption{Facteurs d'ouverture du collimateur (FOC)}
  \label{fig_foc}
\end{figure}

\begin{table}[t!]
  \centering
  \begin{tabular}{>{\centering\arraybackslash}m{1.5cm}>{\centering\arraybackslash}m{1.8cm}>{\centering\arraybackslash}m{1.8cm}>{\centering\arraybackslash}m{1.8cm}>{\centering\arraybackslash}m{1.8cm}>{\centering\arraybackslash}m{1.8cm}>{\centering\arraybackslash}m{1.8cm}}
  \toprule
  \textbf{Champ (cm}$\mathbf{^2}$\textbf{)} &
  \textbf{FOC ref (\%)} &
  \textbf{FOC X23 (\%)} &
  \textbf{FOC Farmer (\%)} &
  \textbf{FOC DSP 80 (\%)} &
  \textbf{FOC DSP 120 (\%)} \\ \toprule
  3x3      & 82,44 & 84,13 & 81,12 & 82,54 & 83,35 \\
  6x6      & 91,71 & 94,38  & 91,74 & 91,97 & 91,88 \\
  8x8      & 96,18 & 97,74 & 96,38 & 96,47  & 96,39 \\
  10x10 & 100      & 100      & 100      & 100      & 100      \\
  12x12 & 102,93 & 101,83 & 102,99 & 102,79 & 102,89 \\
  15x15    & 106,51 & 103,79 & 106,45 & 106,27 & 106,59 \\
  20x20 & 110,94 & 106,06 & 110,55 & 110,45 & 111,08 \\ \bottomrule
  \end{tabular}
  \caption{Résultats des mesures des FOC}
  \label{table_resultats_foc}
\end{table}

\clearpage
\bibliography{biblio}
\addcontentsline{toc}{section}{Références}
\bibliographystyle{plain}
\nocite{*}

\end{document}